\documentclass[a4paper, 12pt]{article}

\usepackage[utf8]{inputenc}
\usepackage[T1, T2A]{fontenc}
\usepackage[english, russian]{babel}
\usepackage{indentfirst}
\usepackage[top=2cm, bottom=2.5cm, left=2cm, right=2cm]{geometry}
\usepackage{graphicx}

\usepackage{pgfplots}
\pgfplotsset{compat=1.13}
\pgfplotsset{grid style={dashed, black}}

\usepackage{subcaption}

\usepackage{amsmath}

\usepackage{caption} 
\captionsetup[figure]{name = Рисунок, labelsep = endash}
\captionsetup[table]{name = Таблица, labelsep = endash, justification=raggedright, singlelinecheck=false}
\renewcommand{\baselinestretch}{1.5} 
\parindent=1.27cm
\usepackage{pdfpages}
\begin{document}
	\includepdf{titul}
	\paragraph*{Цель работы:}Исследование динамических и частотных характеристик, анализ структурных свойств и устойчивости линейных непрерывных систем с помощью прикладного пакета Matlab Control System Toolbox. 
	\paragraph*{Начальные данные:}В качестве объекта исследования выбраны линейные непрерывные динамические стационарные системы. Исходная модель разомкнутой системы представляется форме вход-выход и описывается передаточной функцией вида:
	\begin{equation}
	W(s)=\frac{b_1s+b_0}{s(a_2s^2+a_1s+a_0)}
	\end{equation}\par
	Значения коэффициентов $a_0,a_1,a_2,b_0,b_1$ в числителе и знаменателе передаточной функции для выполнения лабораторной работы выбираются самостоятельно произвольно из условия $a_2\neq0$, $b_1\neq0$.\par
	Выберем коэффициенты следующим образом: $a_0=1,a_1=2,a_2=3,b_0=0,b_1=4$\par
	В итоге получается функция вида:
	\begin{equation}
	W(s)=\frac{4s}{s(s^2+2s+3)}=\frac{4s}{s^3+2s^2+3s}
	\end{equation}
	\newpage
	\begin{center}
		\section{Анализ исходной разомкнутой системы}
	\end{center}
	\subsection{Нули и полюса передаточной функции разомкнутой системы}
	Схему расположения нулей и полюсов можно получить при помощи команды pzmap(w). Полученная схема исследуемой функции приведена на рисунке 1.
	\begin{figure}[h!]
		\begin{center}
			\includegraphics{pzmap}
			\caption{Схема нулей и полюсов функции}
		\end{center}  
	\end{figure}\\
	Полученные полюса:\\
	$s_1=-1 + 1.4142i$\\
	$s_2=-1 - 1.4142i$
	\newpage
	\subsection{Получение графика логарифмических амплитудночастотной и фазочастотной характеристик}
	Графики можно получить при помощи команды margin(w). Полученные графики приведены на рисунке 2.
	\begin{figure}[h!]
		\begin{center}
			\includegraphics{margin}
			\caption{Графики логарифмических АЧХ и ФЧХ}
		\end{center}  
	\end{figure}\par
	По графику видно, что запас устойчивости по амплитуде бесконечный, а по фазе - 78 градусов. 
	\newpage
	\subsection{Построение амплитудно-фазочастотной характеристики исходной системы}
	Построить амплитудно-фазочастотную характеристику можно при помощи команды nyqist(w).
	\begin{figure}[h!]
		\begin{center}
			\includegraphics{nyqist}
			\caption{Фазовый портрет разомкнутой системы}
		\end{center}  
	\end{figure}\par
	По следствию критерия устойчивости Найквиста, система устойчива, так как гадограф разомкнутой системы не охватывает точку [-1;0].
	\newpage
	\begin{center}
	\section{Анализ замкнутой системы}
	\end{center}\par
	Построим замкнутую систему с коэффициентом отрицательно обратной связи 1. Для этого можно воспользоваться командой feedback(w, 1). Получается функция:
	\begin{equation}
	W(s)=\frac{4s}{s(s^2+2s+7)}=\frac{4}{s^2+2s+7}
	\end{equation}
	\subsection{Нули и полюса передаточной функции замкнутой системы} 
	Построим схему расположения нулей и полюсов при помощи команды pzmap(w). Схема представлена на рисунке 4.
	\begin{figure}[h!]
		\begin{center}
			\includegraphics{pzmap1}
			\caption{Схема нулей и полюсов}
		\end{center}  
	\end{figure}\\
	Полученные полюса:\\
	$s_1=-1 + 2.4495i$\\
	$s_2=-1 - 2.4495i$
	\newpage
	\subsection{Получение графика переходной и весовой функций замкнутой системы}
	Для построение графика переходной функции используется команда step(w). Полученный график на рисунке 5.
	\begin{figure}[h!]
		\begin{center}
			\includegraphics{step}
			\caption{График переходного процесса замкнутой системы}
		\end{center}  
	\end{figure}\par
	По графику переходного процесса функции видно, что установившееся значение 0.57, время переходного процесса $t_\text{п}=5.5c$, перерегулирование $\sigma=\frac{0.735-0.57}{0.57}*100\%=29\%$, затухание в данной системе равно нулю. 
	\newpage
	\subsection{Переход к представлению замкнутой системы в форме ВСВ}
	Команда [A,B,C,D]=tf2ss(a,b), где a-числитель, b-знаменатель, позволяет перейти к форме вход-состояние выход.\\
	Полученные матрицы:\\
	$A=\begin{bmatrix} -2 & -7 \\ 1 & 0 \end{bmatrix}$\\
	$B=\begin{bmatrix}  1 \\ 0  \end{bmatrix}$\\
	$C=\begin{bmatrix}  0 & 4  \end{bmatrix}$\\
	\newpage
	\begin{center}
		\section*{Вывод}
	\end{center}\par
	Пакет CST позволяет очень быстро и точно исследовать систему.С его помощью можно получить все характеристики исследуемой системы.\par
	Полученные полюса разомкнутой системы имеют отрицательную вещественную часть, что свидетельствует об устойчивости системы. Это подтверждается фазовым портретом и построенными ЛАЧХ и ЛФЧХ. По ЛАЧХ видно, что запас устойчивости системы по амплитуде бесконечный, что говорит о том, что нельзя подобрать такой коэффициент обратной связи, который вывел бы эту систему из равновесия.\par
	Вещественные части полюсов замкнутой передаточной функции так же имеют отрицательную обратную связь, что говорит об устойчивости системы. Это подтверждает график переходного процесса.
\end{document}